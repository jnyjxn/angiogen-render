% Pour le bon support de la langue française : 
\pdfminorversion=5
\pdfobjcompresslevel=2
\usepackage[T1]{fontenc}
\usepackage[utf8]{inputenc}
\usepackage{lmodern}

\usepackage{etex}
\usepackage{lipsum}

% Packages perso :
\usepackage{soul}
\usepackage{fancyhdr}
\usepackage{boxedminipage}
\usepackage{makeidx}
\makeindex
\usepackage{caption}
\usepackage[final]{pdfpages}
\usepackage{listings} % pour inclure des algorithmes
\newcommand{\deriv}{\mathrm{d}}

\usepackage{color} % définit une nouvelle couleur
\usepackage{xcolor}             % Pour mettre de la couleur
% Bibliographie dans la Table des matières
%\usepackage[nottoc, notlof, notlot]{tocbibind} 
\usepackage{authoraftertitle}   %\MyAuthor

\usepackage{float}              % Pour les figures
\usepackage{pgfplots}
\usepackage{tikz}
\usetikzlibrary{arrows,shapes,positioning}
\usetikzlibrary{decorations.markings,decorations.pathmorphing,
decorations.pathreplacing}
\usetikzlibrary{calc,patterns,shapes.geometric}
\tikzset{every picture/.style={execute at begin picture={
\shorthandoff{:;!?};}
}}
\usepackage[europeancurrents,europeanresistors,cuteinductors,europeanports,europeanvoltages]{circuitikz}         % Circuits électriques
\usepackage{thumbpdf}           % Fichier pdf généré 
                                % contien une              
                                % miniature de chaque slide 
%\usepackage{frbib}              % Bibliographie en fr
\usepackage{dashrule}           % Lignes points-tillés
\usepackage{pifont}             % \ding{code utf8}
\usepackage{hyperref}           % Pour ajouter des liens avec \href
\usepackage{lastpage}           % Avoir total de pages dans le footer.
\usepackage{graphicx}           % Pour les images et figures
\usepackage{wrapfig}            % Pour détourer les figures
%\begin{wrapfigure}[lineheight]{alignement}{width}
\usepackage{multicol}           % Doc sur plusieurs colonnes
\setlength{\columnseprule}{.4pt}% Avec séparateur.
\usepackage{fancybox}           % De chouettes encadrements
\usepackage{lettrine}           % Pour de beaux paragraphes
% \usepackage{type1cm}          % scalable fonts pour lettrine
\usepackage{oldgerm}            % Lettrines en gothique
\newcommand{\gothique}[2]{\lettrine{\textgoth #1}{#2}} 
\usepackage{yfonts}             % De belles enluminures
\newcommand{\enluminure}[2]{\lettrine[lines=3]{\small \initfamily #1}{#2}} 
\usepackage{url}                % Pour écrire des url
\usepackage{geometry}
\geometry{margin=60pt}
\usepackage{setspace}           % Pour changer l'interligne
\usepackage{eurosym}            % Pour le signe \euro
\usepackage{colortbl}           % Couleur dans tableaux
\usepackage{tabularx}           % pour des tableaux à taille de la page
\usepackage{longtable}          % Pour les grands tableaux
\usepackage[tight]{shorttoc}    % Pour faire un sommaire à la française.
\newcommand{\sommaire}{\shorttoc{Sommaire}{1}}
\usepackage{array}              % De beaux tableaux
\usepackage{multirow}           % Tableaux sur plusieurs lignes
                                % \multirow{nlignes}{largeur ou *}{contenu}
\usepackage{amsmath}            % Un peu de maths
\usepackage{amssymb}            % Encore des maths
\usepackage{mathtools}
\usepackage{empheq}             % Pour encadrer les équations
\usepackage{pdfpages}           % Pour include des pdf avec
%\includepdf[pages=-]{votre_fichier} 
% ,pagecommand={}               % Pour aussi mettre le num de la page
\usepackage[normalem]{ulem}               % Pour barrer du texte

\usepackage{pgfpages}           % Pour avoir 2 pages sur A4 paysage
\usepackage{datetime}           % Jouer facilement avec les dates
% Pour la physique : 
\usepackage{numprint}           % Pour faire des groupes de 3 nombres
\usepackage[squaren,Gray,cdot]{SIunits} % Unités SI
% Ecrire des symboles atomiques
\newcommand{\noyau}[3]{\prescript{#2}{#3}{\mathrm{#1}}}
% Utilisation $\noyau{C}{12}{6}$

% Pour la chimie : 
\usepackage[version=3]{mhchem}
\usepackage{chemfig}

%%%
% Commandes

% Ajout d'une image avec label
\newcommand{\image}[3]{
% \image{fichier}{label}{description}
\begin{center}
% Nécessite le package float
\begin{figure}[H]
\includegraphics[width=0.8\textwidth]{#1}
\caption{\label{#2}{#3}}
\end{figure}
\end{center}
}

% Ajout d'une image largeur page avec label
\newcommand{\imagebig}[3]{
% \imagebig{fichier}{label}{description}
\begin{center}
% Nécessite le package float
\begin{figure}[H]
\includegraphics[width=1\textwidth]{#1}
\caption{\label{#2}{#3}}
\end{figure}
\end{center}
}

% Une image qui prend toute la page
\newcommand{\imagefull}[1]{
    \newgeometry{margin=0cm}
\begin{center}
\begin{figure}[H]
\includegraphics[width=0.96\paperwidth]{#1}
\end{figure}
\end{center}
\restoregeometry
\nopagebreak
}

% Un encadré grisé
\newcommand{\encadregris}[1]{
\begin{center}
\colorbox{gray!20}{
\begin{minipage}{0.95\textwidth}
{#1}
\end{minipage}
}
\end{center}
}

% Un mot grisé
\newcommand{\motgris}[1]{
\colorbox{gray!20}{{#1}}
}

% Un encadré
\newcommand{\encadre}[1]{
\begin{center}
\fbox{
\begin{minipage}{0.95\textwidth}
{#1}
\end{minipage}
}
\end{center}
}

% flèche
\newcommand{\ra}[0]{
    $\rightarrow$
}

% Image détourée
% \wrapimg{align}{width}{img}
\newcommand{\wrapimg}[3]{
\begin{wrapfigure}{#1}{#2}
\includegraphics[width={#2}]{#3}
\end{wrapfigure}
}

\newcommand{\hdr}[0]{
    \hdashrule{1cm}{1pt}{1pt}
}

\newcommand{\hdrp}[0]{
    \hdashrule{3cm}{1pt}{1pt}
}

\newcommand{\hdrpp}[0]{
    \hdashrule{4cm}{1pt}{1pt}
}

\newcommand*{\etoile}
{
\begin{center}
*\par
*\hspace*{3ex}*
\end{center}
}

% petit carré
\newcommand{\petitcarre}{\rule[.2ex]{0.75ex}{0.75ex}}

%incertitude relative
\newcommand{\incertRel}[1]{
    \frac{\Delta {#1}}{{#1}}
}

% listes avec puces carrées
\newcommand{\carlst}[1]{
    \begin{itemize}
    \renewcommand\labelitemi{\petitcarre}
    {#1}
    \end{itemize}
}

% \emph plus simple
\newcommand{\e}[1]{\emph{{#1}}}

% equation encadrée
\newcommand{\boxeq}[1]{
    \begin{empheq}[box=\shadowbox]{align}
    {#1}
    \end{empheq}
}

%%% Schémas de chimie %%%

% Schema de titrage
% \schemaTitrage{solution titrante}{solution titrée}
\newcommand{\schemaTitrage}[2]{
    \begin{tikzpicture}
        \draw [thick]
        (0,0) -- ++ (2,0) -- ++ (0,1) -- ++ (-2,0) -- ++ (0,-1)
        (0.3,3) -- ++ (0,-2)
        (1.7,3) -- ++ (0,-2)
        (1,6) -- ++ (0,-3)
        (0.8,3.3) -- ++ (0.4,0)
        (0.3,1.7) -- ++ (1.4,0)
        (0.3,0.3) circle(0.1)
        (1,1.11)ellipse (0.25 and 0.10) 
        ;
        \draw
        [<-,>=stealth] (1,4.4) -- ++ (1.5,0) node[right]  {
            \begin{minipage}{0.275\textwidth}
                {#1}
            \end{minipage}
        }
        ;
        \draw
        [<-,>=stealth] (1,1.5) -- ++ (1.5,0) node[right] {
            \begin{minipage}{0.27\textwidth} 
            {#2}
            \end{minipage}
        }
        ;
    \end{tikzpicture}
}

% Tube à essai légendé.
\newcommand{\schemaTubeEssai}[1]{
    \begin{tikzpicture}[scale=0.25]
    \draw
    (0,0) arc (0:-180:1)
    (-2,0) -- ++ (0,8)
    (0,0) -- ++ (0,8)
    (-2,4) -- ++ (2,0)
    (-0.5,-1) node[below] 
        {
        \begin{minipage}{0.2\textwidth}
        {#1}
        \end{minipage}
        }
    ;
    \end{tikzpicture}
}
% Papier millimétré
% \papmili{x}{y}
\newcommand{\papmili}[2]{
\begin{tikzpicture}
\draw[step=1mm, line width=0.1mm, black!30!white] (0,0) grid
({#1},{#2});
\draw[step=5mm, line width=0.2mm, black!40!white] (0,0) grid
({#1},{#2});
\draw[step=5cm, line width=0.5mm, black!50!white] (0,0) grid
({#1},{#2});
\draw[step=1cm, line width=0.3mm, black!90!white] (0,0) grid
({#1},{#2});
\end{tikzpicture}
}
% \grap{x}{y}{legende x}{legende y}
\newcommand{\graph}[4]{
\begin{tikzpicture}
\draw[step=1mm, line width=0.1mm, black!30!white] (0,0) grid
({#1},{#2});
\draw[step=5mm, line width=0.2mm, black!60!white] (0,0) grid
({#1},{#2});
\draw[step=5cm, line width=0.5mm, black!80!white] (0,0) grid
({#1},{#2});
\draw[line width=0.6mm, ->, >=stealth] (0,0) -- ({#1},0)
node[right]{{#3}};
\draw[line width=0.6mm, ->, >=stealth] (0,0) -- (0,{#2})
node[above]{{#4}};
\end{tikzpicture}
}

\usepackage{listingsutf8}

\lstdefinestyle{customc}{
  belowcaptionskip=1\baselineskip,
  breaklines=true,
  frame=L,
  xleftmargin=\parindent,
  language=C,
  showstringspaces=false,
  basicstyle=\footnotesize\ttfamily,
  keywordstyle=\bfseries\color{green!40!black},
  commentstyle=\itshape\color{purple!40!black},
  identifierstyle=\color{blue},
  stringstyle=\color{orange},
}

\lstdefinestyle{customasm}{
  belowcaptionskip=1\baselineskip,
  frame=L,
  xleftmargin=\parindent,
  language=[x86masm]Assembler,
  basicstyle=\footnotesize\ttfamily,
  commentstyle=\itshape\color{purple!40!black},
}

\lstset{escapechar=@,style=customc}