\documentclass[11pt,oneside,a4paper,final]{article}

\usepackage{ifpdf}
%HEVEA\pdffalse

\ifpdf
\usepackage{fancyhdr}
\usepackage{lastpage}

\pagestyle{fancy}

\renewcommand{\headrulewidth}{0.0pt}

\lhead{}
\chead{}
\rhead{}
\lfoot{}
\cfoot{Page {\thepage} of \pageref{LastPage}}
\rfoot{}
\fi

\usepackage{longtable}

\usepackage[T1]{fontenc}  
\usepackage{textcomp}  
\usepackage{lmodern}  

% \usepackage[style=numeric,backend=biber]{biblatex}
% \usepackage[style=numeric]{biblatex}
\bibliography{programmer_guide}
% \usepackage[hidelinks=true]{hyperref}



\usepackage{hyperref}

\hypersetup{  
  colorlinks=true,
  final=true,
  pdftitle="gVirtualXRay - Tutorial 01 - GLUT",
  pdfauthor="Dr Franck P. Vidal",  
  pdfsubject="Creating a Window and an OpenGL Context Using GLUT"
}

\usepackage[paper=a4paper,margin=1.6cm]{geometry}	% 1 inch margins all around

\usepackage[british=nohyphenation]{hyphsubst}


\usepackage[british]{babel}
\usepackage{graphicx}
\usepackage[font=normalsize,labelfont=sf,textfont=sf]{subfig}
\captionsetup*[subfigure]{position=bottom}
\usepackage{color}

\definecolor{mygreen}{rgb}{0,0.6,0}
\definecolor{mygray}{rgb}{0.5,0.5,0.5}
\definecolor{mymauve}{rgb}{0.58,0,0.82}

\usepackage{listings}
 \lstset{ %
  literate={"}{\textquotedbl}1,
  backgroundcolor=\color{white},   % choose the background color; you must add \usepackage{color} or \usepackage{xcolor}
  basicstyle=\footnotesize,        % the size of the fonts that are used for the code
  breakatwhitespace=false,         % sets if automatic breaks should only happen at whitespace
  breaklines=true,                 % sets automatic line breaking
  captionpos=b,                    % sets the caption-position to bottom
  commentstyle=\color{mygreen},    % comment style
%   deletekeywords={...},            % if you want to delete keywords from the given language
%   escapeinside={\%*}{*)},          % if you want to add LaTeX within your code
  extendedchars=true,              % lets you use non-ASCII characters; for 8-bits encodings only, does not work with UTF-8
  frame=single,                    % adds a frame around the code
  keepspaces=true,                 % keeps spaces in text, useful for keeping indentation of code (possibly needs columns=flexible)
  keywordstyle=\color{blue},       % keyword style
  language=C++,                 % the language of the code
%   morekeywords={*,...},            % if you want to add more keywords to the set
  numbers=left,                    % where to put the line-numbers; possible values are (none, left, right)
  numbersep=5pt,                   % how far the line-numbers are from the code
  numberstyle=\tiny\color{mygray}, % the style that is used for the line-numbers
  rulecolor=\color{black},         % if not set, the frame-color may be changed on line-breaks within not-black text (e.g. comments (green here))
  showspaces=false,                % show spaces everywhere adding particular underscores; it overrides 'showstringspaces'
  showstringspaces=false,          % underline spaces within strings only
  showtabs=false,                  % show tabs within strings adding particular underscores
  stepnumber=1,                    % the step between two line-numbers. If it's 1, each line will be numbered
  stringstyle=\color{mymauve},     % string literal style
  tabsize=2,                       % sets default tabsize to 2 spaces
  title=\lstname                   % show the filename of files included with \lstinputlisting; also try caption instead of title
}

% \usepackage[acronym,toc]{glossaries}
% \makeglossaries
% % \phantomsection
% \addcontentsline{toc}{section}{Acronyms}
% \section*{Acronyms}
% \begin{footnotesize}
% \begin{acronym}[TDMA]

% \newacronym{2D}{2D}{two-dimensional}
\newacronym{3D}{3D}{three-dimensional}
\newacronym{FBO}{FBO}{frame buffer object}
\newacronym{FLTK}{FLTK}{Fast Light Toolkit}
\newacronym{GLEW}{GLEW}{OpenGL Extension Wrangler Library}
\newacronym{GLSL}{GLSL}{OpenGL Shading Language}
\newacronym{GLU}{GLU}{OpenGL Utility Library}
\newacronym{GLUT}{GLUT}{OpenGL Utility Toolkit}
\newacronym{GPU}{GPU}{Graphics Processor Unit}
% \newacronym{GUI}{GUI}{graphical user interface}
% \newacronym{Gzip}{Gzip}{GNU zip}
% \newacronym{ITK}{ITK}{Insight Segmentation and Registration Toolkit}
\newacronym{keV}{keV}{kiloelectron volt}
% \newacronym{MXE}{MXE}{M cross environment}
\newacronym{STL}{STL}{STereoLithography}
\newacronym{SVN}{SVN}{Subversion}
\newacronym{VBO}{VBO}{vertex buffer object}
% \newacronym{VTK}{VTK}{Visualization Toolkit}
% 


\title{gVirtualXRay -- Tutorial 01: Creating a Window and\\an OpenGL Context Using GLUT}

\author{Dr Franck P. Vidal}
\date{4\textsuperscript{th} September 2014}

\begin{document}
 \sloppy

\maketitle
\vfill
\begin{center}
\includegraphics[width=0.5\textwidth]{screenshot}
\end{center}
\vfill


 \newpage
\phantomsection
\addcontentsline{toc}{section}{Table of contents}
\tableofcontents

\ifpdf
\newpage
\phantomsection
\addcontentsline{toc}{section}{List of figures}
\listoffigures

% \newpage
% \phantomsection
% \addcontentsline{toc}{section}{List of tables}
% \listoftables

% \newpage
\phantomsection
\addcontentsline{toc}{section}{List of listings}
\lstlistoflistings
\fi

\newpage


%%%%%%%%%%%%%%%%%%%%%%%%%%%%%%%%%%%%%%%%%%%%%%%%%%%%%%%%%%%%%%%%%%%%%%%%%%%%%%%%
\section{Introduction}

The complete source code of this tutorial is available in 
Appendix~\ref{sec:Program Source Code} and on the Subversion (SVN) repository 
at 
\verb+example/tutorial_01_glut/tutorial_01_glut.cxx+. 
It can be downloaded here: 
\url{
https://sourceforge.net/p/gvirtualxray/code/HEAD/tree/trunk/tutorials/tutorial_01_glut/tutorial_01_glut.cxx}. 
It shows how to create a window with GLUT\,\footnote{\url{http://www.pengl.org/resources/libraries/glut//}} 
and attach an OpenGL context to it. 
Two cubes are displayed (see Figure~\ref{fig:scene}). 
\begin{figure}[tbh]
 \centering
 \includegraphics[width=0.5\textwidth]{screenshot}
 \caption{\label{fig:scene} Screen capture of the tutorial.}
\end{figure}
They both rotate automatically. 
The rendering makes use of OpenGL Shading Language (GLSL) as the fix rendering 
pipeline and direct 
rendering are both depreciated in any modern computer graphics applications. 

It is an introductory tutorial, for more details, the reader may refer to the 
code (it is well documented) and the 
Doxygen\,\footnote{\url{http://www.doxygen.org/}} documentation of the 
project\,\footnote{\url{http://gvirtualxray.sourceforge.net/documentation.php}}
. 

The tutorial is organised as follows:
\begin{itemize}
 \item Section~\ref{sec:Header inclusion} shows the header files to include. 

 \item Section~\ref{sec:Name Spaces} shows some of the name spaces that can be 
	included to lighten the code. 
	% How to set up all the main components of the simulation is describe in 

 \item Section~\ref{sec:Global Variables} shows the global variables that are 
	used. 

 \item Section~\ref{sec:Function Declarations} what typical functions have to 
	be declared in a basic GLUT program. 

 \item Section~\ref{sec:Initialise GLUT} shows how to initialise GLUT. 

 \item Section~\ref{sec:Initialise GLEW} shows how to initialise 
    GLEW\,\footnote{\url{http://glew.sourceforge.net/}}. 

 \item Section~\ref{sec:Initialise OpenGL} shows how to initialise OpenGL. 

 \item How to load 3D objects is explained in Section~\ref{sec:Load Objects} 

 \item They are displayed in Section~\ref{sec:Display Scene}.

 \item Section~\ref{sec:Idle Callback} shows the idle callback. 

 \item Section~\ref{sec:Quit Callback} shows the routine that is called when 
  the program shuts.

 \item Section~\ref{sec:frame buffer callback} shows what is done when the sze 
of the frame buffer changes, i.e.~when the window is resized.

 \item Section~\ref{sec:Keyboard Callback} shows how to handle the keyboard.

 \item Section~\ref{sec:Scroll button callback} shows how to handle the 
  scroll button.
  
 \item Section~\ref{sec:Next Tutorial} gives a preview of what the next 
  tutorial will be about.
 
 \item Appendix~\ref{sec:Program Source Code} shows the source code of this 
	tutorial. 
	
\end{itemize}


%%%%%%%%%%%%%%%%%%%%%%%%%%%%%%%%%%%%%%%%%%%%%%%%%%%%%%%%%%%%%%%%%%%%%%%%%%%%%%%%
\section{Header inclusion}
\label{sec:Header inclusion}

Listing~\ref{lst:header} shows 
 the header files that need to be included to 
build a simple program based on the GLUT 
library:
\begin{itemize}
 \item \verb+GL/glew.h+ is the GLEW header file. It can be found at 
  \url{http://glew.sourceforge.net/}. GLEW is used to ensure that there is no 
  undefined references when the Windows executable is created.
 \item \verb+glut.h+ is the GLUT header file. Be careful here as Apple installed it somewhere else... 
 \item \verb+iostream+ is used for output streams.
 \item \verb+sstream+ is used to generate error messages
 \item \verb+exception+ is used for C++ exceptions.
 \item \verb+cstdlib+ defines return status (\verb+EXIT_SUCCESS+ and 
	\verb+EXIT_FAILURE+).
 \item \verb+gVirtualXRay/Types.h+ defines new types, e.g 
\verb+RATIONAL_NUMBER+, 
	\verb+VEC2+, \verb+VEC3+ and \verb+MATRIX4+ to name a few.
 \item \verb+gVirtualXRay/Units.h+ defines units such as metre, kilometre, 
electronvolt, 
	kiloelectron volt, gram, kilogram, etc.
 \item \verb+gVirtualXRay/OpenGLUtilities.h+ defines some utility functions 
about OpenGL, 
	e.g. matrix stacks, how to set the projection matrix, etc.
 \item \verb+gVirtualXRay/PolygonMesh.h+ corresponds to a class used to handle 
three-dimensional (3D) triangle meshes.
 \item \verb+gVirtualXRay/Shader.h+ corresponds to a class that handles 
(GLSL) programs.
 \item \verb+buildCube.h+ is used to create the triangle mesh of a cube.
\end{itemize}

\begin{center}
\lstinputlisting[firstline=62, lastline=92, caption=Header inclusion., 
label=lst:header]{../../../tutorials/tutorial_01_glut/tutorial_01_glut.cxx}
\end{center}


%%%%%%%%%%%%%%%%%%%%%%%%%%%%%%%%%%%%%%%%%%%%%%%%%%%%%%%%%%%%%%%%%%%%%%%%%%%%%%%%
\section{Name Spaces}
\label{sec:Name Spaces}

Listing~\ref{lst:name space} shows the name spaces that can be selected:
\begin{itemize}
 \item \verb+gVirtualXRay+ includes graphics elements such as \verb+PolygonMesh+ 
	and utilities such as \verb+Exception+. 
 \item \verb+std+ is used for output streams and exceptions.
\end{itemize}

\begin{center}
\lstinputlisting[firstline=95, lastline=99, caption=Name spaces., 
label=lst:name space]{../../../tutorials/tutorial_01_glut/tutorial_01_glut.cxx}
\end{center}


%%%%%%%%%%%%%%%%%%%%%%%%%%%%%%%%%%%%%%%%%%%%%%%%%%%%%%%%%%%%%%%%%%%%%%%%%%%%%%%%
\section{Global Variables}
\label{sec:Global Variables}

Listing~\ref{lst:global variables} shows the global variables that are used:
\begin{itemize}
	\item \verb+GLsizei g_window_width+ keeps track of the window width.

	\item \verb+GLsizei g_window_height+ keeps track of the window height.
	
	\item \verb+int g_window_id+ is the GLUT window ID.
	
	\item \verb+Shader g_display_shader+ is the shader program used to 
		display the 3D scene
	
	\item \verb+PolygonMesh g_polygon_mesh_1+ is the polygon mesh of the first 
		3D object.
	
	\item \verb+PolygonMesh g_polygon_mesh_2+ is the polygon mesh of the second 
		3D object.

	\item \verb+MATRIX4 g_object_1_rotation_matrix+ corresponds to the 
		transformation matrix of the first 3D object.
	
	\item \verb+MATRIX4 g_object_2_rotation_matrix+corresponds to the 
		transformation matrix of the second 3D object.

	\item \verb+vector<double> g_p_vertex_set_1+ is an array containing 
		the vertices of the first 3D object.
		
	\item \verb+vector<unsigned char> g_p_index_set_1+ is an array 
		containing vertex indices to build triangles from 
		\verb+g_p_vertex_set_1+.

	\item \verb+vector<float> g_p_vertex_set_2+ is an array containing 
		the vertices of the second 3D object. Note that no index is used in 
		this case.

	\item \verb+RATIONAL_NUMBER g_zoom+ controls the zoom.

	\item \verb+const GLchar* g_vertex_shader+ is the source code of the vertex 
		shader.

	\item \verb+const GLchar* g_fragment_shader+ is the source code of the 
		fragment shader.
\end{itemize}


\begin{center}
\lstinputlisting[firstline=102, lastline=154, caption=Global variables., 
label=lst:global variables]{../../../tutorials/tutorial_01_glut/tutorial_01_glut.cxx}
\end{center}


%%%%%%%%%%%%%%%%%%%%%%%%%%%%%%%%%%%%%%%%%%%%%%%%%%%%%%%%%%%%%%%%%%%%%%%%%%%%%%%%
\section{Function Declarations}
\label{sec:Function Declarations}

Listing~\ref{lst:function declarations} shows the typical functions 
that have to be declared in a basic GLUT program:
\begin{itemize}
 \item \verb+initGLUT+ is used i) to initialise GLUT, ii) to create an OpenGL 
	context, iii) to create a window, and iv) attach the OpenGL context to the 
	window (see Section~\ref{sec:Initialise GLUT}).
 \item \verb+initGLEW+ is used to initialise GLEW (see 
	Section~\ref{sec:Initialise GLEW}).
 \item \verb+initGL+ is used to initialise some states of OpenGL, e.g. the 
	background colour and enable the Z-buffer (see 
	Section~\ref{sec:Initialise OpenGL}).
 \item \verb+load3DObjects+ loads the 3D geometry of two cubes (see 
	Section~\ref{sec:Load Objects}).
 \item \verb+displayCallback+ render the two cubes on the screen (see 
	Section~\ref{sec:Display Scene}).
 \item \verb+idleCallback+ is an idle callback 
	(see Section~\ref{sec:Idle Callback}). It is called once every event 
	loop and can be used to perform animation.
 \item \verb+quitCallback+ is call when the program terminates  (see 
	Section~\ref{sec:Quit Callback}). It closes the window and cleans up 
	GLUT.
 \item \verb+framebufferSizeCallback+ is called every time the frame buffer 
	size changes (see Section~\ref{sec:frame buffer callback}). It initialises 
	the viewport size and the projection matrix.
 \item \verb+keyCallback+ is called every time a key is pressed or released on 
	the keyboard (see Section~\ref{sec:Keyboard Callback}). It can be used to 
	close the window when the \verb+escape+ key is pressed.
\item \verb+scrollCallback+ processes the mouse scroll button. It can be used 
	to zoom in and out (see Section~\ref{sec:Scroll button callback}).
\end{itemize}

\begin{center}
\lstinputlisting[firstline=157, lastline=169, caption=Function declarations., 
label=lst:function declarations]{../../../tutorials/tutorial_01_glut/tutorial_01_glut.cxx}
\end{center}


%%%%%%%%%%%%%%%%%%%%%%%%%%%%%%%%%%%%%%%%%%%%%%%%%%%%%%%%%%%%%%%%%%%%%%%%%%%%%%%%
\section{Initialise GLUT}
\label{sec:Initialise GLUT}

Listing~\ref{lst:Initialise GLUT} shows how to initialise GLUT to create an 
OpenGL core profile 3.2 context and how to attach a window to it. There are 
nime main steps:
\begin{itemize}
    \item Initialise the GLUT library.
    \item Create a windowed mode window and its OpenGL context
    \item If the window has not been created, an exception is thrown.
    \item Register GLUT callbacks
\end{itemize}

\begin{center}
\lstinputlisting[firstline=223, lastline=248, caption=Initialise GLUT., 
label=lst:Initialise GLUT]{../../../tutorials/tutorial_01_glut/tutorial_01_glut.cxx}
\end{center}


%%%%%%%%%%%%%%%%%%%%%%%%%%%%%%%%%%%%%%%%%%%%%%%%%%%%%%%%%%%%%%%%%%%%%%%%%%%%%%%%
\section{Initialise GLEW}
\label{sec:Initialise GLEW}

Listing~\ref{lst:Initialise GLEW} shows how to initialise GLEW.

\begin{center}
\lstinputlisting[firstline=251, lastline=263, caption=Initialise GLEW., 
label=lst:Initialise GLEW]{../../../tutorials/tutorial_01_glut/tutorial_01_glut.cxx}
\end{center}


%%%%%%%%%%%%%%%%%%%%%%%%%%%%%%%%%%%%%%%%%%%%%%%%%%%%%%%%%%%%%%%%%%%%%%%%%%%%%%%%
\section{Initialise OpenGL}
\label{sec:Initialise OpenGL}

Listing~\ref{lst:Initialise OpenGL} shows how to initialise some OpenGL states 
(Z-buffer, and background colour) and check OpenGL's error status. 

\begin{center}
\lstinputlisting[firstline=266, lastline=279, caption=Initialise OpenGL., 
label=lst:Initialise OpenGL]{../../../tutorials/tutorial_01_glut/tutorial_01_glut.cxx}
\end{center}


%%%%%%%%%%%%%%%%%%%%%%%%%%%%%%%%%%%%%%%%%%%%%%%%%%%%%%%%%%%%%%%%%%%%%%%%%%%%%%%%
\section{Load 3D Objects}
\label{sec:Load Objects}

Listing~\ref{lst:Load_Objects} shows how to create 3D objects. It 
implements 
two cubes. The first one makes use of a vertex list and an index list (see 
Figure~\ref{fig:cube_01}). The second cube only makes use of a vertex list. 
Vertices are repeated several times. This is because each vertex is shared by 
different triangles. The second cube is therefore much bigger in term of 
memory usage. Listing~\ref{lst:Load_Objects} shows that our implementation 
supports both type of topology.

\begin{figure}[tbh]
 \centering
 \ifpdf
  \subfloat[Geometry.]{\includegraphics[width=0.2\textwidth]{cube_01}}
  \subfloat[Vertex list.]{
    \raisebox{3em}{\begin{tabular}{|c|c|c|c|}
	\cline{2-4}
	\multicolumn{1}{c|}{} & \multicolumn{3}{c|}{\textbf{Vertex List}} \\
	\hline
	\textbf{Vertex ID} & \textbf{x} & \textbf{y} & \textbf{z} \\
	\hline
	0 & -2.5 cm &  2.5 cm & -2.5 cm \\
	\hline
	1 & -2.5 cm &  2.5 cm &  2.5 cm \\
	\hline
	2 &  2.5 cm &  2.5 cm & -2.5 cm \\
	\hline
	3 &  2.5 cm &  2.5 cm &  2.5 cm \\
	\hline
	4 & -2.5 cm & -2.5 cm & -2.5 cm \\
	\hline
	5 & -2.5 cm & -2.5 cm &  2.5 cm \\
	\hline
	6 &  2.5 cm & -2.5 cm & -2.5 cm \\
	\hline
	7 &  2.5 cm & -2.5 cm &  2.5 cm \\
	\hline
\end{tabular}}
 }
\subfloat[Index list.]{
\raisebox{1em}{
\begin{tabular}{|c|c|c|c|}
	\cline{2-4}
	\multicolumn{1}{c|}{} & \multicolumn{3}{c|}{\textbf{Face List}} \\
	\cline{2-4}
	\multicolumn{1}{c|}{} & \textbf{i} & \textbf{j} & \textbf{k} \\
	\hline
	Top face & 1 & 2 & 0 \\
	\hline
	Top face & 2 & 1 & 3 \\
	\hline
	Bottom face & 6 & 5 & 4 \\
	\hline
	Bottom face & 5 & 6 & 7 \\
	\hline
	Front face & 5 & 3 & 1 \\
	\hline
	Front face & 3 & 5 & 7 \\
	\hline
	Back face & 2 & 4 & 0 \\
	\hline
	Back face & 4 & 2 & 6 \\
	\hline
	Left face & 5 & 0 & 4 \\
	\hline
	Left face & 0 & 5 & 1 \\
	\hline
	Right face & 2 & 7 & 6 \\
	\hline
	Right face & 7 & 2 & 3 \\
	\hline
\end{tabular}}}
\else
  Geometry.\\{\includegraphics[width=0.2\textwidth]{cube_01}}\\[1em]
  
  Vertex list.\\{
    {\begin{tabular}{|c|c|c|c|}
	\cline{2-4}
	\multicolumn{1}{c|}{} & \multicolumn{3}{c|}{\textbf{Vertex List}} \\
	\hline
	\textbf{Vertex ID} & \textbf{x} & \textbf{y} & \textbf{z} \\
	\hline
	0 & -2.5 cm &  2.5 cm & -2.5 cm \\
	\hline
	1 & -2.5 cm &  2.5 cm &  2.5 cm \\
	\hline
	2 &  2.5 cm &  2.5 cm & -2.5 cm \\
	\hline
	3 &  2.5 cm &  2.5 cm &  2.5 cm \\
	\hline
	4 & -2.5 cm & -2.5 cm & -2.5 cm \\
	\hline
	5 & -2.5 cm & -2.5 cm &  2.5 cm \\
	\hline
	6 &  2.5 cm & -2.5 cm & -2.5 cm \\
	\hline
	7 &  2.5 cm & -2.5 cm &  2.5 cm \\
	\hline
\end{tabular}}
 }\\[1em]

 Index list.\\{
{
\begin{tabular}{|c|c|c|c|}
	\cline{2-4}
	\multicolumn{1}{c|}{} & \multicolumn{3}{c|}{\textbf{Face List}} \\
	\cline{2-4}
	\multicolumn{1}{c|}{} & \textbf{i} & \textbf{j} & \textbf{k} \\
	\hline
	Top face & 1 & 2 & 0 \\
	\hline
	Top face & 2 & 1 & 3 \\
	\hline
	Bottom face & 6 & 5 & 4 \\
	\hline
	Bottom face & 5 & 6 & 7 \\
	\hline
	Front face & 5 & 3 & 1 \\
	\hline
	Front face & 3 & 5 & 7 \\
	\hline
	Back face & 2 & 4 & 0 \\
	\hline
	Back face & 4 & 2 & 6 \\
	\hline
	Left face & 5 & 0 & 4 \\
	\hline
	Left face & 0 & 5 & 1 \\
	\hline
	Right face & 2 & 7 & 6 \\
	\hline
	Right face & 7 & 2 & 3 \\
	\hline
\end{tabular}}}
\fi
 \caption{\label{fig:cube_01} Topology of the first cube.}
\end{figure}

\begin{figure}[tbh]
 \centering
 \ifpdf
 \subfloat[Geometry.]{\includegraphics[width=0.35\textwidth]{cube_02}}
\subfloat[Vertex list.]{
\raisebox{2cm}{\begin{tabular}{|c|c|c|c|}
	\cline{2-4}
	\multicolumn{1}{c|}{} & \multicolumn{3}{c|}{\textbf{Vertex List}} \\
	\hline
	\textbf{Vertex ID} & \textbf{x} & \textbf{y} & \textbf{z} \\
	\hline
% 	// Top face
	 0 & -2.5 cm &  2.5 cm & 2.5 cm \\
	\hline
	 1 & 2.5 cm &  2.5 cm & -2.5 cm \\
	\hline
	 $\vdots$ & $\vdots$ & $\vdots$ & $\vdots$ \\
	\hline
% 	 2 & -2.5 cm &  2.5 cm & -2.5 cm \\
% 	\hline
% 	 3 & 2.5 cm &  2.5 cm & -2.5 cm \\
% 	\hline
% 	 4 & -2.5 cm &  2.5 cm & 2.5 cm \\
% 	\hline
% 	 5 & 2.5 cm &  2.5 cm & 2.5 cm \\
% 	\hline
% % 	// Bottom face
% 	 6 & 2.5 cm & -2.5 cm & -2.5 cm \\
% 	\hline
% 	 7 & -2.5 cm & -2.5 cm &  2.5 cm \\
% 	\hline
% 	 8 & -2.5 cm & -2.5 cm & -2.5 cm \\
% 	\hline
% 	 9 & -2.5 cm & -2.5 cm &  2.5 cm \\
% 	\hline
% 	10 & 2.5 cm & -2.5 cm & -2.5 cm \\
% 	\hline
% 	11 &  2.5 cm & -2.5 cm &  2.5 cm \\
% 	\hline
% % 	// Front face
% 	12 & -2.5 cm & -2.5 cm &  2.5 cm \\
% 	\hline
% 	13 & 2.5 cm &  2.5 cm & 2.5 cm \\
% 	\hline
% 	14 & -2.5 cm &  2.5 cm & 2.5 cm \\
% 	\hline
% 	15 & 2.5 cm &  2.5 cm & 2.5 cm \\
% 	\hline
% 	16 & -2.5 cm & -2.5 cm &  2.5 cm \\
% 	\hline
% 	17 &  2.5 cm & -2.5 cm &  2.5 cm \\
% 	\hline
% % 	// Back face
% 	18 & 2.5 cm &  2.5 cm & -2.5 cm \\
% 	\hline
% 	19 & -2.5 cm & -2.5 cm & -2.5 cm \\
% 	\hline
% 	20 & -2.5 cm &  2.5 cm & -2.5 cm \\
% 	\hline
% 	21 & -2.5 cm & -2.5 cm & -2.5 cm \\
% 	\hline
% 	22 & 2.5 cm &  2.5 cm & -2.5 cm \\
% 	\hline
% 	23 & 2.5 cm & -2.5 cm & -2.5 cm \\
% 	\hline
% % 	// Left face
% 	24 & -2.5 cm & -2.5 cm &  2.5 cm \\
% 	\hline
% 	25 & -2.5 cm &  2.5 cm & -2.5 cm \\
% 	\hline
% 	26 & -2.5 cm & -2.5 cm & -2.5 cm \\
% 	\hline
% 	27 & -2.5 cm &  2.5 cm & -2.5 cm \\
% 	\hline
% 	28 & -2.5 cm & -2.5 cm &  2.5 cm \\
% 	\hline
% 	29 & -2.5 cm &  2.5 cm & 2.5 cm \\
% 	\hline
% % 	// Right face
% 	30 & 2.5 cm &  2.5 cm & -2.5 cm \\
% 	\hline
% 	31 &  2.5 cm & -2.5 cm &  2.5 cm \\
% 	\hline
% 	32 & 2.5 cm & -2.5 cm & -2.5 cm \\
% 	\hline
% 	33 &  2.5 cm & -2.5 cm &  2.5 cm \\
% 	\hline
	34 & 2.5 cm &  2.5 cm & -2.5 cm \\
	\hline
	35 & 2.5 cm &  2.5 cm & 2.5 cm \\
	\hline
\end{tabular}}}
\else
 Geometry.\\{\includegraphics[width=0.35\textwidth]{cube_02}}\\[1em]
 
Vertex list.\\{
{\begin{tabular}{|c|c|c|c|}
	\cline{2-4}
	\multicolumn{1}{c|}{} & \multicolumn{3}{c|}{\textbf{Vertex List}} \\
	\hline
	\textbf{Vertex ID} & \textbf{x} & \textbf{y} & \textbf{z} \\
	\hline
% 	// Top face
	 0 & -2.5 cm &  2.5 cm & 2.5 cm \\
	\hline
	 1 & 2.5 cm &  2.5 cm & -2.5 cm \\
	\hline
	 $\vdots$ & $\vdots$ & $\vdots$ & $\vdots$ \\
	\hline
% 	 2 & -2.5 cm &  2.5 cm & -2.5 cm \\
% 	\hline
% 	 3 & 2.5 cm &  2.5 cm & -2.5 cm \\
% 	\hline
% 	 4 & -2.5 cm &  2.5 cm & 2.5 cm \\
% 	\hline
% 	 5 & 2.5 cm &  2.5 cm & 2.5 cm \\
% 	\hline
% % 	// Bottom face
% 	 6 & 2.5 cm & -2.5 cm & -2.5 cm \\
% 	\hline
% 	 7 & -2.5 cm & -2.5 cm &  2.5 cm \\
% 	\hline
% 	 8 & -2.5 cm & -2.5 cm & -2.5 cm \\
% 	\hline
% 	 9 & -2.5 cm & -2.5 cm &  2.5 cm \\
% 	\hline
% 	10 & 2.5 cm & -2.5 cm & -2.5 cm \\
% 	\hline
% 	11 &  2.5 cm & -2.5 cm &  2.5 cm \\
% 	\hline
% % 	// Front face
% 	12 & -2.5 cm & -2.5 cm &  2.5 cm \\
% 	\hline
% 	13 & 2.5 cm &  2.5 cm & 2.5 cm \\
% 	\hline
% 	14 & -2.5 cm &  2.5 cm & 2.5 cm \\
% 	\hline
% 	15 & 2.5 cm &  2.5 cm & 2.5 cm \\
% 	\hline
% 	16 & -2.5 cm & -2.5 cm &  2.5 cm \\
% 	\hline
% 	17 &  2.5 cm & -2.5 cm &  2.5 cm \\
% 	\hline
% % 	// Back face
% 	18 & 2.5 cm &  2.5 cm & -2.5 cm \\
% 	\hline
% 	19 & -2.5 cm & -2.5 cm & -2.5 cm \\
% 	\hline
% 	20 & -2.5 cm &  2.5 cm & -2.5 cm \\
% 	\hline
% 	21 & -2.5 cm & -2.5 cm & -2.5 cm \\
% 	\hline
% 	22 & 2.5 cm &  2.5 cm & -2.5 cm \\
% 	\hline
% 	23 & 2.5 cm & -2.5 cm & -2.5 cm \\
% 	\hline
% % 	// Left face
% 	24 & -2.5 cm & -2.5 cm &  2.5 cm \\
% 	\hline
% 	25 & -2.5 cm &  2.5 cm & -2.5 cm \\
% 	\hline
% 	26 & -2.5 cm & -2.5 cm & -2.5 cm \\
% 	\hline
% 	27 & -2.5 cm &  2.5 cm & -2.5 cm \\
% 	\hline
% 	28 & -2.5 cm & -2.5 cm &  2.5 cm \\
% 	\hline
% 	29 & -2.5 cm &  2.5 cm & 2.5 cm \\
% 	\hline
% % 	// Right face
% 	30 & 2.5 cm &  2.5 cm & -2.5 cm \\
% 	\hline
% 	31 &  2.5 cm & -2.5 cm &  2.5 cm \\
% 	\hline
% 	32 & 2.5 cm & -2.5 cm & -2.5 cm \\
% 	\hline
% 	33 &  2.5 cm & -2.5 cm &  2.5 cm \\
% 	\hline
	34 & 2.5 cm &  2.5 cm & -2.5 cm \\
	\hline
	35 & 2.5 cm &  2.5 cm & 2.5 cm \\
	\hline
\end{tabular}}}
\fi
 \caption{\label{fig:cube_02} Topology of the second cube.}
\end{figure}

\begin{itemize}
  \item \verb+buildCube(length_1, cube_centre, g_p_vertex_set_1, +
    \verb+g_p_index_set_1)+ creates the data required to model a cube using 
    vertex data and index data. Its length is \verb+length_1+ and it is centred 
    on \verb+cube_centre+ (0, 0, 0).

  \item \verb+buildCube(length_2, cube_centre, g_p_vertex_set_2)+ creates the 
    data of another cube, but using vertex data only. Its length is 
    \verb+length_2+ and it is also centred on 
    \verb+cube_centre+.

  \item \verb+g_polygon_mesh_1.setExternalData(GL_TRIANGLES,+ 
    \verb+ &g_p_vertex_set_1, &g_p_index_set_1, +\\ \verb+true, +
    \verb+GL_STATIC_DRAW)+ loads both the vertex (\verb+&g_p_vertex_set_1+) and 
    index (\verb+&g_p_index_set_1+) data of the geometry. 
    \verb+GL_TRIANGLES+ indicates that the mesh corresponds to a triangle mesh. 
    \verb+true+ means that the vertex buffer object (VBO) should be 
created instantly if 
    possible. 
    \verb+GL_STATIC_DRAW+ is used because the (VBO) data will be set once 
    and used many times.
     
  \item \verb+g_polygon_mesh_2.setExternalData(GL_TRIANGLES,+ 
    \verb+ &g_p_vertex_set_2, true, +\\
    \verb+GL_STATIC_DRAW)+ loads the vertex (\verb+&g_p_vertex_set_2+) data of 
    the geometry. There is no index data for this mesh. 
\end{itemize}

\begin{center}
\lstinputlisting[label=lst:Load_Objects,
    firstline=282, lastline=310, 
    caption=Create 3D objects.]{../../../tutorials/tutorial_01_glut/tutorial_01_glut.cxx}
\end{center}


%%%%%%%%%%%%%%%%%%%%%%%%%%%%%%%%%%%%%%%%%%%%%%%%%%%%%%%%%%%%%%%%%%%%%%%%%%%%%%%%
\section{Display the 3D Scene}
\label{sec:Display Scene}

Listing~\ref{lst:Display_Scene} shows how to display the 3D objects. 
\begin{itemize}
 \item 
    // Clear the buffers
    \verb+glClear(GL_COLOR_BUFFER_BIT | GL_DEPTH_BUFFER_BIT)+ clears both the 
      fragment and Z buffers.

 \item \verb+pushShaderProgram()+ adds the current shader program to the shader 
    stack. It may become handy when multiple shaders are used.

 \item \verb+g_display_shader.enable()+ enables the given shader program.

 \item Its unique OpenGL ID can be retrieved with 
    \verb+g_display_shader.getProgramHandle()+.

 \item The status of the frame buffer object (FBO) can be checked with 
    \verb+checkFBOErrorStatus(__FILE__, __FUNCTION__, __LINE__)+. 
    If an error had occurred, then an exception will be thrown.

 \item The status of the OpenGL's error flag can be checked with 
    \verb+checkOpenGLErrorStatus(__FILE__, __FUNCTION__, __LINE__)+. 
    If an error had occurred, then an exception will be thrown.

 \item \verb+g_current_projection_matrix+ is the current projection matrix. 
 \verb+g_current_modelview_matrix+ is the current modelling-viewing matrix. 
  Section~\ref{sec:frame buffer callback} shows how it can be set. 
  Using a shader program, when the fixed pipeline functions are disabled (as in 
  modern OpenGL implementation), the programmer has to supply the projection 
  and modelling-viewing matrices to the shader program. This is what Lines~22 
to~30 are taking care of in Listing~\ref{lst:Display_Scene}.

 \item To store the current transformation matrices, 
    \verb+pushModelViewMatrix()+ and \verb+pushProjectionMatrix()+ are used. 
    They replace the old \verb+glMatrixMode+ and \verb+glPushMatrix+, which are 
    no longer available in modern OpenGL.

  \item To translate the 1\textsuperscript{st} object, write
    \verb+g_current_modelview_matrix *=+\\
    \verb+MATRIX4::buildTranslationMatrix(VEC3(8.0 * cm, 0.0, 0.0))+. This is 
    similar to the old function \verb+glTranslate+, which is now depreciated. 

   \item To rotate this object, write 
     \verb+g_current_modelview_matrix *= g_object_1_rotation_matrix+. This is 
    similar to the old function \verb+glRotate+, which is now depreciated. 

  \item To apply the changes to the shader program, call 
    \verb+applyModelViewMatrix()+.
    
  \item The 1\textsuperscript{st} polygon mesh is displayed with 
    \verb+g_polygon_mesh_1.display()+.

  \item The transformation matrices are restored from the stack using 
    \verb+popModelViewMatrix()+ and \verb+popProjectionMatrix()+.

  \item The transformation matrices are stored in the stack using 
    \verb+pushModelViewMatrix()+ and \verb+pushProjectionMatrix()+.

  \item To translate the 2\textsuperscript{nd} object, write
    \verb+g_current_modelview_matrix *=+\\
    \verb+MATRIX4::buildTranslationMatrix(VEC3(-8.0 * cm, 0.0, 0.0))+.

   \item To rotate this object, write 
     \verb+g_current_modelview_matrix *= g_object_2_rotation_matrix+.

  \item To apply the changes to the shader program, call 
    \verb+applyModelViewMatrix()+.

  \item The 2\textsuperscript{nd} object is displayed by 
    \verb+g_polygon_mesh_2.display()+.

  \item The transformation matrices are restored from the stack using 
    \verb+popModelViewMatrix()+ and \verb+popProjectionMatrix()+.

  \item \verb+popShaderProgram()+ disables the current shader and restores the 
    previous shader from the stack.

  \item Finally, we check the error status of both OpenGL and the (FBO).

\end{itemize}


\begin{center}
\lstinputlisting[label=lst:Display_Scene,
    firstline=313, lastline=395,
    caption=display the 3D scene.]{../../../tutorials/tutorial_01_glut/tutorial_01_glut.cxx}
\end{center}


%%%%%%%%%%%%%%%%%%%%%%%%%%%%%%%%%%%%%%%%%%%%%%%%%%%%%%%%%%%%%%%%%%%%%%%%%%%%%%%%
\section{Idle Callback}
\label{sec:Idle Callback}

In the idle callback the rotation matrices are updated to create an animation 
(see Listing~\ref{lst:Idle callback}). 
\verb+g_object_1_rotation_matrix.rotate(1.0, VEC3(1.0, 0.0, 0.0))+ multiplies 
the transformation matrix (\verb+g_object_1_rotation_matrix+) by a rotation 
matrix defined by angle in degrees (\verb+1.0+ in this example) and a rotation 
axis (\verb+VEC3(1.0, 0.0, 0.0)+).

\begin{center}
\lstinputlisting[firstline=399, lastline=408, caption=Idle callback., 
label=lst:Idle callback]{../../../tutorials/tutorial_01_glut/tutorial_01_glut.cxx}
\end{center}

	
%%%%%%%%%%%%%%%%%%%%%%%%%%%%%%%%%%%%%%%%%%%%%%%%%%%%%%%%%%%%%%%%%%%%%%%%%%%%%%%%
\section{Quit Callback}
\label{sec:Quit Callback}

Listing~\ref{lst:Quit callback} shows how to clean up GLUT. 
If the window exists, then it is destroyed. 
Finally, the GLUT application is terminated. 

\begin{center}
\lstinputlisting[firstline=411, lastline=422, caption=Quit callback., 
label=lst:Quit callback]{../../../tutorials/tutorial_01_glut/tutorial_01_glut.cxx}
\end{center}


%%%%%%%%%%%%%%%%%%%%%%%%%%%%%%%%%%%%%%%%%%%%%%%%%%%%%%%%%%%%%%%%%%%%%%%%%%%%%%%%
\section{Frame Buffer Size Callback}
\label{sec:frame buffer callback}

This callback is called when the size of the frame buffer changes, that is to 
say when the size of the window changes. In this function three main things are 
performed (see Listing~\ref{lst:frame buffer callback}).
\begin{itemize}
 \item Set the OpenGL's viewport with \verb+glViewport+.
 
 \item Set the projection matrix. In this example; we use 
    \verb+loadPerspectiveProjectionMatrix+ to set 
    \verb+g_current_projection_matrix+. It is similar to \verb+gluPerspective+, 
    which is not available in modern OpenGL.
 
 \item Set the modelling-viewing transformation. In this example; we use 
    \verb+loadLookAtModelViewMatrix+ to set 
    \verb+g_current_modelview_matrix+. It is similar to \verb+gluLookAt+, 
    which is not available in modern OpenGL.
 
\end{itemize}

\begin{center}
\lstinputlisting[firstline=425, lastline=458, caption=Change of frame buffer 
size callback., 
label=lst:frame buffer callback]{../../../tutorials/tutorial_01_glut/tutorial_01_glut.cxx}
\end{center}


%%%%%%%%%%%%%%%%%%%%%%%%%%%%%%%%%%%%%%%%%%%%%%%%%%%%%%%%%%%%%%%%%%%%%%%%%%%%%%%%
\section{Keyboard Callback}
\label{sec:Keyboard Callback}

Listing~\ref{lst:Keyboard Callback} shows a typical GLUT keyboard callback. 
Here the window is closed when the user presses \verb+Q+ or \verb+Esc+.

\begin{center}
\lstinputlisting[firstline=461, lastline=480, caption=Keyboard callback., 
label=lst:Keyboard Callback]{../../../tutorials/tutorial_01_glut/tutorial_01_glut.cxx}
\end{center}


%%%%%%%%%%%%%%%%%%%%%%%%%%%%%%%%%%%%%%%%%%%%%%%%%%%%%%%%%%%%%%%%%%%%%%%%%%%%%%%%
\section{Scroll Button Callback}
\label{sec:Scroll button callback}

The zoom is updated with the mouse wheel 
(see Listing~\ref{lst:Scroll button callback}). 
To update the modelling-viewing matrix, the Frame Buffer Size Callback is 
called.

\begin{center}
\lstinputlisting[firstline=482, lastline=511, caption=Scroll button callback., 
label=lst:Scroll button callback]{../../../tutorials/tutorial_01_glut/tutorial_01_glut.cxx}
\end{center}


%%%%%%%%%%%%%%%%%%%%%%%%%%%%%%%%%%%%%%%%%%%%%%%%%%%%%%%%%%%%%%%%%%%%%%%%%%%%%%%%
\section{Next Tutorial...}
\label{sec:Next Tutorial}

In the next tutorial:
\begin{itemize}
	\item Load the shader from a file compressed using 
		the Zlib\,\footnote{\url{http://www.zlib.net/}}
%  \item  You will see how to get rid of OpenGL's fixed pipeline (which is now 
% depreciated). 
%  
%  \item You will see how to get rid of OpenGL's matrix stack (which is now 
% depreciated).
% 
	\item We will see how to create an efficient mouse control to turn the 
		3D scene. 
% 
%  \item You will also see how to update the topology of the triangle meshes. 
%  This is important in medical simulations with soft tissue deformations. 
\end{itemize}


% % \newpage
% \printglossary[type=\acronymtype] 

\appendix
\section{Program Source Code}
\label{sec:Program Source Code}

\begin{center}
  \lstinputlisting[caption=All the source code of this tutorial., 
  label=lst:Program Source Code,numbers=none]{../../../tutorials/tutorial_01_glut/tutorial_01_glut.cxx}
\end{center}

% 
% \newpage
% \phantomsection
% \addcontentsline{toc}{section}{References}
% \printbibliography
% 

\end{document}
